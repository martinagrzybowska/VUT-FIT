\documentclass[a4paper,11pt, titlepage]{article}
\usepackage[left=2cm, text={17cm, 24cm}, top=3cm]{geometry}
\usepackage{times}
\usepackage[czech]{babel}
\usepackage[utf8]{inputenc}
\usepackage[T1]{fontenc}
\bibliographystyle{czplain}
\newcommand{\myuv}[1]{\quotedblbase #1\textquotedblleft}

\pagestyle{empty}
\begin{document}

%~^~%~^~%~^~%~%~^~%~%~^~%~% TITLE PAGE ~^~%~%~^~%~%~^~%~%~^~%~%~^~
	
	\begin{titlepage}
		\begin{center}
			{\Huge\textsc{Vysoké učení technické v~Brně}}\\
			\vspace{+0.8em}
			{\huge\textsc{Fakulta informačních technologií}}\\
			\bigskip
			\vspace{\stretch{0.382}} 
			\LARGE{Typografie a~publikování\,--\,4.projekt}\\
			\vspace{-0.25em}
			\Huge{Bibliografické citace}
			\vspace{\stretch{0.618}}
		\end{center}
		{\Large \today \hfill Martina Grzybowská}
	\end{titlepage}

%~^~%~^~%~^~%~%~^~%~%~^~%~% END OF TITLE PAGE ~^~%~%~^~%~%~^~%~%~^~%~%~^~
\newpage
\pagestyle{plain}

\section{Čo je to typografia}
Typografia je forma jazyka \cite{Lupton:Thinking_with_type}. Je tým, čo dáva textu vizuálnu podobu. Vhodným výberom písma môžeme u čitateľa dosiahnuť neutrálny efekt alebo vzbudiť jeho emócie: navodiť napríklad umelecký, politický či filozofický dojem \cite{Ambrose:Typografie}. Tvary a~farby písmen možno využiť na zapôsobenie na istú cieľovú skupinu. Napríklad prírodnej poéme najlepšie pristane ľahká kurzíva a nič nepodporí politické prehlásenie ráznejšie než ťažká egyptienka. Písmo s~oblými tvarmi a~menším duktusom zobrazené v~jemných farbách vyjadrí ženskejší štýl. Mužne zas pôsobí hranatejšie písmo s~hrubšími ťahmi a~tmavými, sýtejšími farbami \cite{Saltz:Zaklady_typografie}.

\section{História typografie}
História západnej typografie sa datuje od vynálezu kníhtlače v~Európe. Teda okolo roku 1450, kedy v~Mohuči Johannes Gutenberg uviedol do
prevádzky prvý tlačiarenský lis využívajúci pohyblivé litery. Ďalších niekoľko storočí dochádzalo predovšetkým k skvalitňovaniu tlačiarenských
strojov vývojom nových písiem. Ďalší výrazný posun v~tlači je zaznamenaný v 19. storočí, kedy prichádza mnoho nových technológií ako 
litografia, farebná tlač, \myuv{rotačka} a~ofsetová tlač \cite{Jirasek:bakalarka}.

\subsection{Etymológia názvu} 
Slovo typografia je odvodené z~gréckeho \emph{typós} (znaky) a~ \emph{graphein} (písať). Znaky boli za rozkvetu Rímskej ríše tesané do kameňa. Slovo \emph{typos} má aj~tento význam \cite{Olsak:Typografie_co_to_je}.

\section{Pravidlá typografie}
Pre tvorbu typografických diel sa vyvinulo mnoho pravidiel. K~tým základným patrí optický stred, zlatý rez, normalizované formáty 
a~iné \cite{Culakova:bakalarka}. Aj~keď poprední dizajnéri často tvrdia, že pravidlá tvorby existujú preto, aby sa porušovali, jedno pravidlo by malo zostať 
zachované vždy - je ním čitateľnosť \cite{Samara:Graficky_design}. Dobrá typografia totiž nie je vidieť, je nerušivým nástrojom k odovzdaniu obsahu informácie. Čitateľ nesmie byť vyčerpávaný ani rušený \cite{Olsak:Typografie_co_to_je}. Základnou chybou sádzača\,--\,amatéra býva snaha dostať toho na stránku čo najviac bez ohľadu na čitateľnosť. Je dosť obtiažne vyjadriť správny postup vedúci k optimálnej sadzbe kvantitatívne \cite{Vesely:Exkurze_do_taju}. Nepísaná múdrosť profesionálnych technických redaktorov (upravovateľov kníh a~iných tlačových materiálov) o\-pie\-ra\-jú\-ca sa o~štúdiu čitateľnosti hovorí, že tlačený text má obsahovať v~riadku priemerne desať alebo menej slov \cite{Durst:Vytvareni_rejstriku}.

\section{Latex}
{\LaTeX} je vysoko kvalitný typografický systém určený pre profesionálne a~poloprofesionálne sádzanie dokumentov \cite{Martinek:Latexove_speciality}. Bol vyvinutý v roku 1985 Leslie Lamportom. {\LaTeX} využíva ako formátovací jazyk sádzací systém {\TeX} - 
značkovací jazyk vyvinutý Donaldom Knuthom v 70. rokoch. Je založený na myšlenienke, že autor dokumentu by sa mal starať len o text článku, zatiaľ čo o~formátovánie sa postarajú vývojári dokumentu \cite{Mittelbach:An_Introduction_to_Latex}. Dnes je už {\TeX} považovaný za ukončený projekt vzhľadom na to, že sa jeho tvorca rozhodol ukončiť ďalší vývoj, dôsledkom čoho sa už žiadne iné programy nesmú nazývať \TeX om \cite{Syropoulos:Zbornik}.

\newpage
\bibliography{literatura}
\end{document}
